\documentclass[10pt,a4paper,onecolumn,notitlepage]{scrartcl}

%%  Os pacotes da AMS devem ser carregados antes de fontenc e babel (Dica do Uma não tão pequena introdução ao LATEX 2ε)
\usepackage{amsmath}  
\usepackage{amsfonts}
\usepackage{amssymb}

\usepackage[utf8x]{inputenc} % compilando com o lualatex não precisa desta linha. lualatex é unicode por padrão.
\usepackage[brazil]{babel}
\usepackage[T1]{fontenc} % sem esse pacote a hifenização não funciona corretamente com palavras acentuadas.
\usepackage{ucs}
\usepackage{graphicx}
%\usepackage{wrapfig}
\usepackage{mathpazo}
\usepackage{hyperref}
%\usepackage{layouts}
%\usepackage{scrpage2}
\usepackage{tikz}
\usepackage{multicol}
\usepackage{siunitx}
\usepackage{enumitem}
\usepackage{float}
\usepackage{fancyhdr}

\usepackage[textheight=24cm]{geometry}

%\renewcommand{\thesection}{\arabic{section}}
%\numberwithin{equation}{section}
\newcommand{\refeq}[1]{Equação \eqref{#1}}
\newcommand{\HRule}{\rule{\linewidth}{0.5mm}}

\author{}
\title{Prática 1: Paquímetro}


%\thispagestyle{myheadings}
\pagestyle{fancy}
\lhead{}
\chead{}
\rhead{\textsc{Pratica 2: Micrômetro}}
\cfoot{\thepage}
\renewcommand{\headrulewidth}{0.4pt}

%\setcounter{tocdepth}{1} % Profundidade do sumário: 0 - mostra apenas os capítulos.

\begin{document}
\thispagestyle{myheadings}

\begin{figure}
\begin{minipage}{0.08\linewidth}
\includegraphics[scale=0.5]{figuras/brasaoUFC.jpg} 
\end{minipage}
\begin{minipage}{0.91\linewidth}
\textsc{Universidade Federal do Ceará}

Disciplina: EM0016 - Física Experimental para Engenharia
\end{minipage}

\begin{minipage}{\linewidth}
%\vspace{0.1cm}
\centering
\textsc{Pratica 2: Micrômetro}
\\
\hrulefill % usar este comando quando não estiver dentro de uma tabela.
\end{minipage}
\end{figure}

\section{Objetivos}

\begin{itemize}
\item Conhecimento do micrômetro e familiarização com seu uso.
\end{itemize}

\section{Material}

\setlist[1]{itemsep=-5pt}
\begin{multicols}{3}
\begin{itemize}
\item Micrômetro
\item Esfera
\item Tarugo
\item Folha de papel
\item Lâmina de barbear
\item Fios
\end{itemize}
\end{multicols}

\section{Fundamentos}
O micrômetro é um instrumento de medida portátil de boa precisão quando que pode ser usado para medir as dimensões de um objeto. A sua faixa de medição está normalmente restrita a 25 milímetros(mais ou menos uma polega) o que restringe bastante seu uso. A Figura \ref{fig:micro} mostra as várias partes que compõem um micrômetro.

\begin{figure}[ht]
\centering
\includegraphics[scale=0.2]{figuras/micro/micro.png} 
\caption{Micrômetro}
\label{fig:micro}
\end{figure}

\setlist[1]{itemsep=-5pt}
\begin{multicols}{2}
\begin{enumerate}
\item Faces Medição
\item Batente
\item Fuso
\item Bainha
\item Bucha Interna
\item Porca de Ajuste
\item Arco
\item Isolante Térmico
\item Trava
\item Linha de Referência
\item Tambor
\item Catraca
\end{enumerate}
\end{multicols}

\subsection{Princípio de Funcionamento do Micrômetro}
O princípio de funcionamento do micrômetro é semelhante ao sistema parafuso porca como pode ser visto na Figura \ref{fig:passo}. Cada volta do parafuso corresponde ao avanço de um passo, dividindo-se a cabeça do parafuso pode-se avaliar frações menores que as dos passo do parafuso.

\begin{figure}[H]
\centering
\includegraphics[scale=0.15]{figuras/micro/passo.png} 
\caption{Passo do micrômetro}
\label{fig:passo}
\end{figure}

\subsection{Resolução do Micrômetro}
A resolução nos micrômetros pode ser de $0,01mm$; $0,001mm$; $0.001"$ ou $0.0001"$. No micrômetro de $0$ a $25mm$ ou de $0$ a $1"$, quando as faces dos contatos estão juntas, a borda do tambor coincide com o traço zero (0) da bainha. A linha longitudinal, gravada na bainha, coincide com o zero (0) da escala do tambor.

A cada volta do tambor, o fuso do micrômetro avança a distância de um passo e a resolução de um micrômetro corresponde ao menor deslocamento de seu fuso. Para obter a medida, divide-se o passo pelo número de divisões do tambor.


\[
\text{Resolução} = \frac{\text{passo da rosca do fuso micrométrico}}{\text{número de divisões do tambor}}
\]

\textbf{Exemplo:} Para um micrômetro com passo da rosca igual a $0,5mm$ e número de divisões do tambor igual $50$ temos:

\[
\text{Resolução} = \frac{\text{passo da rosca do fuso micrométrico}}{\text{número de divisões do tambor}}=\frac{0,5mm}{50}= 0,01mm
\]
 
\subsection{Medindo com um micrômetro}

A leitura dos valores em um micrômetro consiste basicamente de três passos:
\begin{description}
\item[1o. passo -] leitura dos milímetros inteiros na escala da bainha.
\item[2o. passo -] leitura dos meios milímetros, também na escala da bainha.
\item[3o. passo -] leitura dos centésimos de milímetro na escala do tambor.
\end{description}
Ao final soma-se os valores encontrados em cada um dos passos indicados. Veja o exemplo na Figura \ref{fig:leitura} para a leitura em um micrômetro com resolução de $0,01mm$.


\begin{figure}[H]
\centering
\includegraphics[scale=0.3]{figuras/micro/leitura.png} 
\caption{Leitura em um micrômetro com resolução de $0,01mm$}
\label{fig:leitura}
\end{figure}

Na Figura \ref{fig:leitura} temos:
\[
\begin{matrix}
& 17,00mm & \text{(escala dos milímetros da bainha)}\\
+ & 0,50mm & \text{(escala dos meio milímetros da bainha)}\\
& 0,32mm & \text{(escala centesimal do tambor)}\\
\hline
& 17,82mm
\end{matrix}
\]

\textbf{OBS.:} Em geral não se usa estimar os milésimos de milímetro, entretanto para medidas menores do que $0,10mm$ isto se faz necessário para evitar que a medida fique com somente um algarismo significativo.

\section{Pré-laboratório}
Faça as leituras das figuras abaixo.

\begin{figure}[H]
\label{fig:prelab}
\centering
\begin{minipage}{\linewidth}
\begin{minipage}{0.5\linewidth}
\includegraphics[scale=0.2]{figuras/micro/ex1.png} 
\end{minipage}
\begin{minipage}{0.5\linewidth}
\includegraphics[scale=0.2]{figuras/micro/ex2.png} 
\end{minipage}
\begin{minipage}{0.4\linewidth}
\vspace{0.5cm}
Leitura:\rule{3cm}{0.4pt}
\vspace{1cm}
\end{minipage}
\begin{minipage}{0.4\linewidth}
\vspace{0.5cm}
Leitura:\rule{3cm}{0.4pt}
\vspace{1cm}
\end{minipage}
\begin{minipage}{0.5\linewidth}
\includegraphics[scale=0.2]{figuras/micro/ex3.png} 
\end{minipage}
\begin{minipage}{0.5\linewidth}
\includegraphics[scale=0.2]{figuras/micro/ex4.png} 
\end{minipage}
\begin{minipage}{0.5\linewidth}
\vspace{0.5cm}
Leitura:\rule{3cm}{0.4pt}
\vspace{1cm}
\end{minipage}
\begin{minipage}{0.5\linewidth}
\vspace{0.5cm}
Leitura:\rule{3cm}{0.4pt}
\vspace{1cm}
\end{minipage}
\end{minipage}
\end{figure}


\section{Procedimento}
OBS.: Antes de você fazer esta prática é conveniente conhecer o conteúdo do texto sobre \emph{Algarismos Significativos}. O aluno que não observar as regras sobre Algarismos Significativos em seus relatórios será penalizado.

Utilizando o cálculo do \emph{valor médio}, em que o número de termos  é o mesmo dos números componentes da equipe, como uso do paquímetro, determine:

\subsection{Cálulos de Volume, diâmetros e espessuras}
\subsubsection{O volume da esfera em $mm^3$. Utilize os cálculos dos valores médios obtidos.} 
\begin{table}[H]
\centering
\begin{tabular}{|l|*{4}{c|}}
\hline & Medida 1& Medida 2& Medida 3& Média \\
\hline Diâmetro(mm)& & &  & \\ 
\hline
\end{tabular}
\end{table}

\begin{table}[H]
\centering
\begin{tabular}{|p{10cm}|}
\hline Cálculo do Volume \\ 
\\
\\
\\
\hline
\end{tabular}
\end{table}

\subsubsection{Calcule,  da  mesma  maneira,  a  área  das  seções  retas  dos  fios  apresentados}
\begin{table}[H]
\centering
\begin{tabular}{|l|*{4}{c|}}
\hline & Medida 1& Medida 2& Medida 3& Média \\
\hline Diâmetro do fio 1(mm)& & &  & \\ 
\hline Diâmetro do fio 2(mm)& & &  & \\ 
\hline Diâmetro do fio 3(mm)& & &  & \\ 
\hline
\end{tabular}
\end{table}

\begin{table}[H]
\centering
\begin{tabular}{|p{10cm}|}
\hline Cálculo das seções retas \\ 
\\
\\
\\
\hline
\end{tabular}
\end{table}

\subsubsection{Meça o diâmetro do tarugo} 
\begin{table}[H]
\centering
\begin{tabular}{|l|*{4}{c|}}
\hline & Medida 1& Medida 2& Medida 3& Média \\
\hline Diâmetro(mm)& & &  & \\ 
\hline
\end{tabular}
\end{table}

\subsubsection{Meça a espessura de um fio de cabelo} 
\begin{table}[H]
\centering
\begin{tabular}{|l|*{4}{c|}}
\hline & Medida 1& Medida 2& Medida 3& Média \\
\hline Espessura(mm)& & &  & \\ 
\hline
\end{tabular}
\end{table}

\subsubsection{Meça a espessura desta folha de papel} 
\begin{table}[H]
\centering
\begin{tabular}{|l|*{4}{c|}}
\hline & Medida 1& Medida 2& Medida 3& Média \\
\hline Espessura(mm)& & &  & \\ 
\hline
\end{tabular}
\end{table}

\subsubsection{Meça a espessura de uma lâmina de barbear} 
\begin{table}[H]
\centering
\begin{tabular}{|l|*{4}{c|}}
\hline & Medida 1& Medida 2& Medida 3& Média \\
\hline Espessura(mm)& & &  & \\ 
\hline
\end{tabular}
\end{table}

\section{Questionário}
\begin{enumerate}
\item Qual  o  instrumento  de  maior  precisão:  o  paquímetro utilizado na Prática 1 ou  o  micrômetro desta prática?  Justifique.
\item Indique  algum  outro  método  que  também  permita  determinar  o  volume  da  esfera.  (Tema  livre).
\item De um modo geral, ao medir com um micrômetro,  quais  as  causas  mais  prováveis  de  erro?
\item Determine  a  precisão  de  um  micrômetro  cujas  características  são:  tambor  dividido  em  50  partes  iguais  e  passo  de  $0,25mm$.
\end{enumerate}

\end{document}