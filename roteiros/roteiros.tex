\documentclass[10pt,a4paper,onecolumn,notitlepage]{scrbook}

%%  Os pacotes da AMS devem ser carregados antes de fontenc e babel (Dica do Uma não tão pequena introdução ao LATEX 2ε)
\usepackage{amsmath}  
\usepackage{amsfonts}
\usepackage{amssymb}

%\usepackage[utf8x]{inputenc} % compilando com o lualatex não precisa desta linha. lualatex é unicode por padrão.
\usepackage[brazil]{babel}
\usepackage[T1]{fontenc} % sem esse pacote a hifenização não funciona corretamente com palavras acentuadas.
\usepackage{ucs}
%\usepackage{draftwatermark}\SetWatermarkScale{1}
\usepackage{graphicx}
%\usepackage{wrapfig}
\usepackage{mathpazo}
\usepackage{fancyhdr}
\usepackage[Sonny]{fncychap}
\usepackage{hyperref}
%\usepackage{layouts}
%\usepackage{scrpage2}
\usepackage{tikz}
%\usepackage{sagetex}
\usepackage{multicol}

\renewcommand{\thesection}{\arabic{section}}
%\numberwithin{equation}{section}
\newcommand{\refeq}[1]{Equação \eqref{#1}}
\newcommand{\HRule}{\rule{\linewidth}{0.5mm}}

\author{Mário Pacheco}
\title{Roteiros dos Experimentos de Laboratório de Física Básica}

\setcounter{tocdepth}{0} % Profundidade do sumário: 0 - mostra apenas os capítulos.

\pagestyle{fancy}
\makeatletter
\renewcommand{\@chapapp}{Prática}
\makeatother

%\usepackage{graphics}
%\usepackage{titlesec}
%\titleformat{\chapter}[display]
%  {\normalfont\Large\raggedleft}
%  {\MakeUppercase{\chaptertitlename}%
%    \rlap{ \resizebox{!}{1.5cm}{\thechapter} \rule{5cm}{1.5cm}}}
%  {10pt}{\Huge}
%\titlespacing*{\chapter}{0pt}{30pt}{20pt}

\begin{document}

%
\begin{titlepage}

\begin{center}


% Upper part of the page
\includegraphics[width=0.15\textwidth]{brasaoUFC.jpg}\\[1cm]    

\textsc{\LARGE Universidade Federal do Ceará}\\[0.5cm]

\textsc{\LARGE Campus do Cariri}\\[1.5cm]

\textsc{\Large 2013}\\[0.5cm]


% Title
\HRule \\[0.4cm]
{ \huge \bfseries Roteiros das Práticas \\[0.2cm] de Laboratório de Física}\\[0.4cm]

\HRule \\[1.5cm]

% Author and supervisor
\begin{minipage}{0.4\textwidth}
\begin{flushleft} \large
\emph{Autores:}\\
%\hline
%\vspace{3mm}
Mário \textsc{Pacheco}\\
João \textsc{Hermínio}\\
Antônio Carlos \textsc{Alonge}\\
Felipe \textsc{Cavalcante}

\end{flushleft}
\end{minipage}
\begin{minipage}{0.4\textwidth}
\begin{flushright} \large
\end{flushright}
\end{minipage}

\vfill

% Bottom of the page
{\large \today}

\end{center}

\end{titlepage}



\tableofcontents
%\listoffigures
%\listoftables

\chapter*{Introdução}
A disciplina de laboratório de física IV tem como objetivo abordar tópicos experimentais relacionados à
disciplina de física IV. Nessa disciplina o estudante tem os primeiros contatos com experiências relacionadas ao
estudo de correntes alternadas, óptica  e física moderna. Na  medida do possível,  as experiências seguem a
mesma ordem da disciplina teórica de física IV. Espera-se com isso, que o estudante tenha a oportunidade de
entender o fenômeno físico do ponto  de vista teórico e experimental. A preparação  dos relatórios de cada
experiência deverá  seguir um padrão que permita  ao estudante entender o desenvolvimento do  método
científico. 

A disciplina de laboratório de física IV é uma matéria experimental, na qual a turma de estudantes se divide em 
grupos de trabalho. No início  de  cada aula, o  professor apresenta uma breve discussão teórica  sobre a
experiência que será realizada. Nessa discussão, os grupos também são orientados na seqüência lógica do 
procedimento experimental. Sugere-se que uma experiência completa deve ser executada em cada aula.  


\documentclass[10pt,a4paper,onecolumn,notitlepage]{scrartcl}

%%  Os pacotes da AMS devem ser carregados antes de fontenc e babel (Dica do Uma não tão pequena introdução ao LATEX 2ε)
\usepackage{amsmath}  
\usepackage{amsfonts}
\usepackage{amssymb}

\usepackage[utf8x]{inputenc} % compilando com o lualatex não precisa desta linha. lualatex é unicode por padrão.
\usepackage[brazil]{babel}
\usepackage[T1]{fontenc} % sem esse pacote a hifenização não funciona corretamente com palavras acentuadas.
\usepackage{ucs}
\usepackage{graphicx}
%\usepackage{wrapfig}
\usepackage{mathpazo}
\usepackage{hyperref}
%\usepackage{layouts}
%\usepackage{scrpage2}
\usepackage{tikz}
\usepackage{multicol}
\usepackage{siunitx}
\usepackage{enumitem}
\usepackage{float}
\usepackage{fancyhdr}

\usepackage[textheight=24cm]{geometry}

%\renewcommand{\thesection}{\arabic{section}}
%\numberwithin{equation}{section}
\newcommand{\refeq}[1]{Equação \eqref{#1}}
\newcommand{\HRule}{\rule{\linewidth}{0.5mm}}

\author{}
\title{Prática 1: Paquímetro}


%\thispagestyle{myheadings}
\pagestyle{fancy}
\lhead{}
\chead{}
\rhead{\textsc{Pratica 1: Paquímetro}}
\cfoot{\thepage}
\renewcommand{\headrulewidth}{0.4pt}

%\setcounter{tocdepth}{1} % Profundidade do sumário: 0 - mostra apenas os capítulos.

\begin{document}
\thispagestyle{myheadings}

\begin{figure}
\begin{minipage}{0.08\linewidth}
\includegraphics[scale=0.5]{figuras/brasaoUFC.jpg} 
\end{minipage}
\begin{minipage}{0.91\linewidth}
\textsc{Universidade Federal do Ceará}

Disciplina: EM0016 - Física Experimental para Engenharia
\end{minipage}

\begin{minipage}{\linewidth}
%\vspace{0.1cm}
\centering
\textsc{Algarismos Significativos e Erros}
\\
\hrulefill % usar este comando quando não estiver dentro de uma tabela.
\end{minipage}
\end{figure}

\section{Algarismos Significativos e Erros}
Medir uma grandeza significa compará-la com outra de mesma espécie e verificar quantas vezes a primeira é menor ou maior do que esta.

Em geral, a precisão de uma medida é determinada pelo instrumento através do qual a medida é realizada e pela habilidade da pessoa que a realiza. Ao fazermos uma medida, devemos expressá-la de maneira que o resultado represente o melhor possível a grandeza medida. Por exemplo, ao medirmos o comprimento mostrado na Figura 1 com uma régua graduada em centímetros verificamos que o mesmo tem com certeza mais de $14cm$. Podemos estimar também que além dos $14cm$ temos mais uns $3mm$. Dizemos, então que o comprimento médido é $14,3cm$. Observe que nesta medida os algarismos 1 e 4 são exatos enquanto que o 3 foi estimado, sendo, portanto um algarismo duvidoso. Por que, então não expressamos o comprimento somente com  os algarismos corretos? A resposta é que $14,3$ dá uma melhor ideia do comprimento medido do que simplesmente $14cm$. Temos, então, 3 algarismos significativos.

Na Figura 2 podemos dizer que o valor medido é $14,35cm$, sendo os algarismos $1,4$ e $3$ todos corretos e o algarismo $5$ estimado. Neste caso temos uma medida com quatro algarismos significativos.

\textbf{Em uma medida, chamamos de algarismos significativos, todos os algarismos corretos mais o primeiro duvidoso.}

O algarismo duvidoso surge sempre ao estimarmos uma fração da menor divisão da escala do aparelho de medida.

\subsection{Operações com Algarismos Significativos}

\subsubsection{Soma e/ou Subtração}
\begin{itemize}
\item Algarismo correto $\pm$ algarismo correto = algarismo correto
\item Algarismo correto $\pm$ algarismo duvidoso = algarismo duvidoso
\item Algarismo duvidoso $\pm$ algarismo duvidoso = algarismo duvidoso
\end{itemize} 

Exemplo 1: Os lados de um triângulo foram medidos por instrumentos diferentes. Obteve-se os seguintes valores: $15,31cm$, $8,752cm$ e $17,7cm$. Calcule o perímetro.

\[
\begin{array}{rrlll}
& 15, & \!\!\!\!\!\!3\underline{1} &\\
+ & 8, & \!\!\!\!\!\!75\underline{2} & \text{* Os algarismos sublinhados são os algarismos duvidosos.}\\
& 17, & \!\!\!\!\!\!\underline{7}  & \\
\hline
& 41, & \!\!\!\!\!\!\underline{762} &
\end{array}
\]

O resultado deve conter apenas uma algarismo duvidoso; portanto $41,8cm$. (Observer a regra do arredondamento: \textbf{Se o algarismo à direita do menor Algarismo Significativo na resposta final é 4 ou menor, o valor é arredondado para baixo. Se o algarismo à direita do menor Algarismo Significativo na resposta final é 5 ou maior, o valor é arredondado para cima}).

O menor algarismo significativo de um número é aquele mais à direita.

Exemplo 2: Subtraia $46,7g$ de $96g$.

\[
\begin{array}{rrlll}
& 9\underline{6} &  &\\
+ & 46, & \!\!\!\!\!\!\underline{7} & \text{* Os algarismos sublinhados são os algarismos duvidosos.}\\
\hline
& 4\underline{9}, & \!\!\!\!\!\!\underline{3} &
\end{array}
\]



\end{document}
\input{paquimetro}
\chapter[Análise Gráfica e MRUA]{Análise Gráfica e Movimento Retilíneo Uniformemente Acelerado}

\section{Introdução}
Um movimento retilíneo chama-se uniformemente acelerado quando a a\-ce\-le\-ra\-ção instantânea é constante (independente do tempo). Isto é,

\begin{equation}
\label{eq:aceleracao}
\dfrac{dv}{dt} = \dfrac{d^{2}x}{dt^2} =  a = constante
\end{equation}

Da \refeq{eq:aceleracao} na página \pageref{eq:aceleracao} podemos obter a equação horária da velocidade, que é dada por:

\begin{equation}
\label{eq:velocidade}
v(t) - v(t_0) = \int_{t_0}^{t}\!\!\!adt = a(t-t_0)
\end{equation}

O valor $v(t) = v(t_0)$ da velocidade no ins\-tan\-te inicial chama-se \emph{velocidade inicial}. Assim, $v(t) = v_0 + a(t-t_0)$ mostrando que a velocidade é uma função linear do tempo no movimento uniformemente acelerado.

Podemos obter a lei horária da posição integrando a equação da velocidade em função do tempo (\refeq{eq:velocidade}).

\begin{equation}
x(t) - x(t_0) = \int_{t_0}^{t}v(t')\!\!\!dt' = v_0(t - t_0) + \frac{1}{2}a(t - t_0)^2
\end{equation}

Se definirmos $x(t_0) = x_0$ como posição inicial. Obtemos, desta forma:

\begin{equation}
x(t) = x(t_0) + v_0(t-t_0) + \frac{1}{2}a(t - t_0)^2
\end{equation}

Também podemos exprimir a velocidade do movimento uniformemente acelerado em função da posição por $v^2 = v_{0}^2 + 2a(t-t_0)^2$; também conhecidad como equaçõa de Torricelli.

Aa esquaç~oes cima descrevem apenas a cinemática do movimento uniformemente acelerado, sem ter a preocupação de descrver a origem destes moviemntos - que é o objeto de estudo da dinâmica,cujos princípios básicos  forma formulador por Galileu e Newton.
\section{Parte Experimental}

\subsection{Objetivo}
analisar o movimento de um objeto sob a ação de uma força constante. Utilizar também a análise gráfica para descrever este movimento e determinar sua aceleração.

\subsection{Material Utilizado}

Talvez esta seja a primeira vez que você lida com um trilho de ar, assim, algumas notas de cuidado são úteis. O trilho possui pequenos orifícios pelos quais ar é expelido sob pressão. O carro que corre sobre o trilho tem o formato de um Y invertido, e se mantém flutuando sobre o colchão de ar formado entre o trilho e o carro pelo ar expelido nos cilindros. Assim, é eseencial manter os orifícios e a superfície do carro limpos e livre de arranhões. Evite, portanto, escrever ou marcar o trilho de ar para não obstruir os orifícios e causar variações no colchão de ar formado.

Importante: Não empurre o carrinho sobre o trilho quando a fonte de ar comprimido estiver ligada. Do contrário, tanto o carrinho quanto o trilho poderão sofrer arranhões.

O trilho de ar possui uma escala milimetrada que pode ser usada para registrar a posição do carro, e dispões de um cronômetro digital para registrar os intervalos de tempo. É mais simples com este equipamento medir o tempo transcorrido em função da distância a ser percorrida, embora posteriormente você possa inverter a dependência e analisar a posição em função do tempo transcorrido.

Decrição do material:

\begin{itemize}
\item 01 trilho de 12cm;
\item cronômetro digital multifunções com fonte DV 12V;
\end{itemize}


\subsection{Procedimentos}

\section{Sugestão para condução da análise dos dados:}



\chapter{Referências}

Anote aqui todas as referência que utilizar e depois passamos para o formato bibtex.



\end{document}