\documentclass[10pt,a4paper,onecolumn,notitlepage]{scrbook}

%%  Os pacotes da AMS devem ser carregados antes de fontenc e babel (Dica do Uma não tão pequena introdução ao LATEX 2ε)
\usepackage{amsmath}  
\usepackage{amsfonts}
\usepackage{amssymb}

%\usepackage[utf8x]{inputenc} % compilando com o lualatex não precisa desta linha. lualatex é unicode por padrão.
\usepackage[brazil]{babel}
\usepackage[T1]{fontenc} % sem esse pacote a hifenização não funciona corretamente com palavras acentuadas.
\usepackage{ucs}
%\usepackage{draftwatermark}\SetWatermarkScale{1}
\usepackage{graphicx}
%\usepackage{wrapfig}
\usepackage{mathpazo}
\usepackage{fancyhdr}
\usepackage[Sonny]{fncychap}
\usepackage{hyperref}
%\usepackage{layouts}
%\usepackage{scrpage2}
\usepackage{tikz}
%\usepackage{sagetex}
\usepackage{multicol}

\renewcommand{\thesection}{\arabic{section}}
%\numberwithin{equation}{section}
\newcommand{\refeq}[1]{Equação \eqref{#1}}
\newcommand{\HRule}{\rule{\linewidth}{0.5mm}}

\author{Mário Pacheco}
\title{Roteiros dos Experimentos de Laboratório de Física Básica}

\setcounter{tocdepth}{0} % Profundidade do sumário: 0 - mostra apenas os capítulos.

\pagestyle{fancy}
\makeatletter
\renewcommand{\@chapapp}{Prática}
\makeatother

%\usepackage{graphics}
%\usepackage{titlesec}
%\titleformat{\chapter}[display]
%  {\normalfont\Large\raggedleft}
%  {\MakeUppercase{\chaptertitlename}%
%    \rlap{ \resizebox{!}{1.5cm}{\thechapter} \rule{5cm}{1.5cm}}}
%  {10pt}{\Huge}
%\titlespacing*{\chapter}{0pt}{30pt}{20pt}

\begin{document}

%
\begin{titlepage}

\begin{center}


% Upper part of the page
\includegraphics[width=0.15\textwidth]{brasaoUFC.jpg}\\[1cm]    

\textsc{\LARGE Universidade Federal do Ceará}\\[0.5cm]

\textsc{\LARGE Campus do Cariri}\\[1.5cm]

\textsc{\Large 2013}\\[0.5cm]


% Title
\HRule \\[0.4cm]
{ \huge \bfseries Roteiros das Práticas \\[0.2cm] de Laboratório de Física}\\[0.4cm]

\HRule \\[1.5cm]

% Author and supervisor
\begin{minipage}{0.4\textwidth}
\begin{flushleft} \large
\emph{Autores:}\\
%\hline
%\vspace{3mm}
Mário \textsc{Pacheco}\\
João \textsc{Hermínio}\\
Antônio Carlos \textsc{Alonge}\\
Felipe \textsc{Cavalcante}

\end{flushleft}
\end{minipage}
\begin{minipage}{0.4\textwidth}
\begin{flushright} \large
\end{flushright}
\end{minipage}

\vfill

% Bottom of the page
{\large \today}

\end{center}

\end{titlepage}



\tableofcontents
%\listoffigures
%\listoftables

\chapter*{Introdução}
A disciplina de laboratório de física IV tem como objetivo abordar tópicos experimentais relacionados à
disciplina de física IV. Nessa disciplina o estudante tem os primeiros contatos com experiências relacionadas ao
estudo de correntes alternadas, óptica  e física moderna. Na  medida do possível,  as experiências seguem a
mesma ordem da disciplina teórica de física IV. Espera-se com isso, que o estudante tenha a oportunidade de
entender o fenômeno físico do ponto  de vista teórico e experimental. A preparação  dos relatórios de cada
experiência deverá  seguir um padrão que permita  ao estudante entender o desenvolvimento do  método
científico. 

A disciplina de laboratório de física IV é uma matéria experimental, na qual a turma de estudantes se divide em 
grupos de trabalho. No início  de  cada aula, o  professor apresenta uma breve discussão teórica  sobre a
experiência que será realizada. Nessa discussão, os grupos também são orientados na seqüência lógica do 
procedimento experimental. Sugere-se que uma experiência completa deve ser executada em cada aula.  


\documentclass[10pt,a4paper,onecolumn,notitlepage]{scrartcl}

%%  Os pacotes da AMS devem ser carregados antes de fontenc e babel (Dica do Uma não tão pequena introdução ao LATEX 2ε)
\usepackage{amsmath}  
\usepackage{amsfonts}
\usepackage{amssymb}

\usepackage[utf8x]{inputenc} % compilando com o lualatex não precisa desta linha. lualatex é unicode por padrão.
\usepackage[brazil]{babel}
\usepackage[T1]{fontenc} % sem esse pacote a hifenização não funciona corretamente com palavras acentuadas.
\usepackage{ucs}
\usepackage{graphicx}
%\usepackage{wrapfig}
\usepackage{mathpazo}
\usepackage{hyperref}
%\usepackage{layouts}
%\usepackage{scrpage2}
\usepackage{tikz}
\usepackage{multicol}
\usepackage{siunitx}
\usepackage{enumitem}
\usepackage{float}
\usepackage{fancyhdr}

\usepackage[textheight=24cm]{geometry}

%\renewcommand{\thesection}{\arabic{section}}
%\numberwithin{equation}{section}
\newcommand{\refeq}[1]{Equação \eqref{#1}}
\newcommand{\HRule}{\rule{\linewidth}{0.5mm}}

\author{}
\title{Prática 1: Paquímetro}


%\thispagestyle{myheadings}
\pagestyle{fancy}
\lhead{}
\chead{}
\rhead{\textsc{Pratica 1: Paquímetro}}
\cfoot{\thepage}
\renewcommand{\headrulewidth}{0.4pt}

%\setcounter{tocdepth}{1} % Profundidade do sumário: 0 - mostra apenas os capítulos.

\begin{document}
\thispagestyle{myheadings}

\begin{figure}
\begin{minipage}{0.08\linewidth}
\includegraphics[scale=0.5]{figuras/brasaoUFC.jpg} 
\end{minipage}
\begin{minipage}{0.91\linewidth}
\textsc{Universidade Federal do Ceará}

Disciplina: EM0016 - Física Experimental para Engenharia
\end{minipage}

\begin{minipage}{\linewidth}
%\vspace{0.1cm}
\centering
\textsc{Algarismos Significativos e Erros}
\\
\hrulefill % usar este comando quando não estiver dentro de uma tabela.
\end{minipage}
\end{figure}

\section{Algarismos Significativos e Erros}
Medir uma grandeza significa compará-la com outra de mesma espécie e verificar quantas vezes a primeira é menor ou maior do que esta.

Em geral, a precisão de uma medida é determinada pelo instrumento através do qual a medida é realizada e pela habilidade da pessoa que a realiza. Ao fazermos uma medida, devemos expressá-la de maneira que o resultado represente o melhor possível a grandeza medida. Por exemplo, ao medirmos o comprimento mostrado na Figura 1 com uma régua graduada em centímetros verificamos que o mesmo tem com certeza mais de $14cm$. Podemos estimar também que além dos $14cm$ temos mais uns $3mm$. Dizemos, então que o comprimento médido é $14,3cm$. Observe que nesta medida os algarismos 1 e 4 são exatos enquanto que o 3 foi estimado, sendo, portanto um algarismo duvidoso. Por que, então não expressamos o comprimento somente com  os algarismos corretos? A resposta é que $14,3$ dá uma melhor ideia do comprimento medido do que simplesmente $14cm$. Temos, então, 3 algarismos significativos.

Na Figura 2 podemos dizer que o valor medido é $14,35cm$, sendo os algarismos $1,4$ e $3$ todos corretos e o algarismo $5$ estimado. Neste caso temos uma medida com quatro algarismos significativos.

\textbf{Em uma medida, chamamos de algarismos significativos, todos os algarismos corretos mais o primeiro duvidoso.}

O algarismo duvidoso surge sempre ao estimarmos uma fração da menor divisão da escala do aparelho de medida.

\subsection{Operações com Algarismos Significativos}

\subsubsection{Soma e/ou Subtração}
\begin{itemize}
\item Algarismo correto $\pm$ algarismo correto = algarismo correto
\item Algarismo correto $\pm$ algarismo duvidoso = algarismo duvidoso
\item Algarismo duvidoso $\pm$ algarismo duvidoso = algarismo duvidoso
\end{itemize} 

Exemplo 1: Os lados de um triângulo foram medidos por instrumentos diferentes. Obteve-se os seguintes valores: $15,31cm$, $8,752cm$ e $17,7cm$. Calcule o perímetro.

\[
\begin{array}{rrlll}
& 15, & \!\!\!\!\!\!3\underline{1} &\\
+ & 8, & \!\!\!\!\!\!75\underline{2} & \text{* Os algarismos sublinhados são os algarismos duvidosos.}\\
& 17, & \!\!\!\!\!\!\underline{7}  & \\
\hline
& 41, & \!\!\!\!\!\!\underline{762} &
\end{array}
\]

O resultado deve conter apenas uma algarismo duvidoso; portanto $41,8cm$. (Observer a regra do arredondamento: \textbf{Se o algarismo à direita do menor Algarismo Significativo na resposta final é 4 ou menor, o valor é arredondado para baixo. Se o algarismo à direita do menor Algarismo Significativo na resposta final é 5 ou maior, o valor é arredondado para cima}).

O menor algarismo significativo de um número é aquele mais à direita.

Exemplo 2: Subtraia $46,7g$ de $96g$.

\[
\begin{array}{rrlll}
& 9\underline{6} &  &\\
+ & 46, & \!\!\!\!\!\!\underline{7} & \text{* Os algarismos sublinhados são os algarismos duvidosos.}\\
\hline
& 4\underline{9}, & \!\!\!\!\!\!\underline{3} &
\end{array}
\]



\end{document}
%\chapter{Paquímetro}

\section{Objetivos}

\begin{itemize}
\item Conhecimento do paquímetro e familiarização com seu uso.
\end{itemize}

\section{Material}

\begin{multicols}{2}
\begin{itemize}
\item Paquímetro
\item Cilindro
\item Tarugo
\item Peça com furo cego
\item Régua
\item Tiras de papel
\end{itemize}
\end{multicols}

\section{Fundamentos}
O paquímetro, também conhecido como calibre, é um instrumento de precisão muito usado em oficinas e laboratórios para: medidas de comprimentos, diâmetros de tarugos, diâmetro interno e externo de tubos, profundidades de furos, transformação de polegadas em milímetros e vice-versa. Consta o paquímetro de uma régua \textbf{A}, Figura \ref{fig:paq} à qual estão solidárias uma mandíbula \textbf{B} e uma orelha \textbf{C}.


\begin{figure}[h]
\label{fig:paq}
\includegraphics[scale=0.4]{figuras/paquimetro.jpg} 
\caption{Paquímetro}
\end{figure}

A régua é geralmente graduada em polegadas (na parte superior) e em milímetros (na parte inferior). Ao longo da régua pode deslizar um cursor \textbf{D} no qual estão estampadas duas reguetas \textbf{E/F}, chamadas nônio (ou vernier). O cursor tem um prologamento superior, a oreglha \textbf{G}, um prolongamento inferior, a mandíbula móvel \textbf{H}, o impulsor \textbf{I}, além de estar fixo a uma haste \textbf{J}. A peça mais importante do paquímetro é o nônio, a qual merece um estudo à parte.

\begin{description}
\item[{\textsc Nônio}] É uma pequena régua cujas características determinam o grau de precisão do paquímetro. O nônio permite fazer, com extatidão, leituras de frações de milímetro. Pode ser construído com uma precisão maior ou menor, como  $\frac{1}{10}mm$, $\frac{1}{50}mm$ e até  $\frac{1}{100}mm$. O princípio da construção do nônio é basicamente o seguinte: ``x'' milímetos da régua princicpal constituem o seu comprimento, o qual é dividido em ``n'' partes.
\end{description}

\begin{figure}[h]
\label{fig:reg-non}
\centering
\def\esqnonio{-0.1cm}
\def\largnonio{6*0.19cm}
%\begin{tikzpicture}[y=.2cm, x=0.05*\textwidth,font=\sffamily]
\begin{tikzpicture}[scale=3]
%régua
    \draw (0,0) -- (2cm,0);
    
    \foreach \x in {0,...,1}
        \draw (\x*1cm,0pt) -- (\x*1cm,15pt)
        node[anchor=south] {$ $};

        \draw (0,15pt) node[anchor=south] {$0$};
        \draw (1cm,15pt) node[anchor=south] {$1cm$};


    \foreach \x in {1,...,15}
        \draw (\x*0.1cm,0pt) -- (\x*0.1cm,5pt)
        node[anchor=south] {};

    \foreach \x in {1,3}
        \draw (\x*0.5cm,0pt) -- (\x*0.5cm,10pt)
        node[anchor=south] {};

    \draw (1.5cm,0.4cm) node[anchor=south] {RÉGUA};

%Nônio
	\draw [gray, thick, fill=lightgray] (\esqnonio, -0.02cm) rectangle (\esqnonio + \largnonio,-0.5cm);
      
    \foreach \x in {0,...,10}
        \draw (\x*0.09cm, -0.01cm) -- (\x*0.09cm, -0.15cm)
        node[anchor=north] {{\tiny \x}};

    \draw (8*0.09cm, -0.45cm) node[anchor=south] {{{\small NÔNIO}}};

\end{tikzpicture}
\\

\caption{Régua e Nônio}
\end{figure}

No caso da Figura \ref{fig:reg-non}, o comprimento do nônio é $9mm$ e foi dividido em 10 partes iguais. Portanto, cada divisão desse nônio é igual a $9/10mm$. Se o traço 0(zero) do nônio está em coincidência com o traço 0 da régua, isto significa que o traço 1 do nônio está afastado $1/10$ do traço de $1mm$ da régua. Por outro lado, se o traço 1 do nônio coincidisse com o traço $1mm$ da régua, o nônio teria sido deslocado $1/10mm$. O mesmo raciocício é válido para os demais traços, como por exemplo: no caso de o traço 6 do nônio coincidir com o traço de $6mm$ da régua, é porque houve um deslocamento do nônio equivalente a $6/10mm$.

\textsc{\textbf{Precisão do Nônio}} - Para encontrar o grau de precisão de um nônio:
\begin{enumerate}
\item Mede-se o comprimento (L) do nônio (a distância entre o primeiro e o último traço);
\item Divide-se o comprimento (L) por (n), que é o número de divisões do nônio;
\item Sutrai-se o resultado do número inteiro de milímetro imediatamente superior.
\end{enumerate}

\begin{figure}[h]
\label{fig:reg-non2}
\centering
\def\esqnonio{-0.1cm}
\def\largnonio{12*0.19cm}
%\begin{tikzpicture}[y=.2cm, x=0.05*\textwidth,font=\sffamily]
\begin{tikzpicture}[scale=3]
%régua
    \draw (0,0) -- (3cm,0);
    
    \foreach \x in {0,...,2}
        \draw (\x*1cm,0pt) -- (\x*1cm,15pt)
        node[anchor=south] {$ $};

    \draw (0,15pt)
        node[anchor=south] {$0$};

    \foreach \x in {1,...,2}
        \draw (\x*1cm,15pt) node[anchor=south] {$\x cm$};


    \foreach \x in {1,...,29}
        \draw (\x*0.1cm,0pt) -- (\x*0.1cm,5pt)
        node[anchor=south] {};

    \foreach \x in {1,3,5}
        \draw (\x*0.5cm,0pt) -- (\x*0.5cm,10pt)
        node[anchor=south] {};

    \draw (2.5cm,0.4cm) node[anchor=south] {RÉGUA};

%Nônio
	\draw [gray, ultra thick, fill=lightgray] (\esqnonio, -0.02cm) rectangle (\esqnonio + \largnonio,-0.5cm);
      
    \foreach \x in {0,2,4,6,8,10,12,14,16,18,20}
        \draw (\x*0.09cm, -0.01cm) -- (\x*0.09cm, -0.15cm)
        node[anchor=north] {{\tiny \x}};

    \foreach \x in {0,1,3,5,7,9,11,13,15,17,19}
        \draw (\x*0.09cm, -0.01cm) -- (\x*0.09cm, -0.10cm)
        node[anchor=south] {};       

    \draw (18*0.09cm, -0.45cm) node[anchor=south] {{{\small NÔNIO}}};
    
\end{tikzpicture}
\caption{Exemplo de Nônio}
\end{figure}

Para o Nõnio da Figura \ref{fig:reg-non2}, temos:
\begin{enumerate}
\item $L = 19mm$;
\item $n=20; 19mm\div 20= 0,95mm$;
\item Precisão = $1mm - 0,95mm = 0,5mm = 1/20mm$.
\end{enumerate}

\textsc{\textbf{Medindo com o Paquímetro}}:
\begin{enumerate}
\item Encoste a peça a medir na mandíbula fixa;
\item Com o polegar no impulsor, desloque o mandíbula móvel até que ela encoste suavemente na outra extremidade da peça;
\item Leia na régua principal o número de milímetros inteiros, ou seja, os que estão à esquerda do zero do nônio;
\item Para a leitura da fração de milímetros, veja qual o traço do nônio que coincide com \textsc{qualquer} traço da régua principal, e multiplique o número desse traço pela precisão;
\item A figura abaixo dá uma ideia de como utilizar as diversas parte do paquímetro.

\end{enumerate}

XXXXXXXXXXXXXXXXXXXXXXX
XXXXXXXXXXXXXXXXXXXXXXX

\section{Pré-laboratório}
Determine a precisão do nônio ilustrado abaixo e faça as leituras das figuras subsequentes.

% % % % % % % % % % % % % % % % % %
\def\esqnonio{-0.1cm}
\def\largnonio{12*0.19cm}

\begin{minipage}{\linewidth}
\begin{minipage}{0.4\linewidth}

%\begin{tikzpicture}[y=.2cm, x=0.05*\textwidth,font=\sffamily]
\begin{tikzpicture}
%régua
    \draw (0,0) -- (3cm,0);
    
    \foreach \x in {0,...,2}
        \draw (\x*1cm,0pt) -- (\x*1cm,15pt)
        node[anchor=south] {$\x$};

    \foreach \x in {1,...,29}
        \draw (\x*0.1cm,0pt) -- (\x*0.1cm,5pt)
        node[anchor=south] {};

    \foreach \x in {1,3,5}
        \draw (\x*0.5cm,0pt) -- (\x*0.5cm,10pt)
        node[anchor=south] {};

%Nônio
	\draw [gray, ultra thick, fill=lightgray] (\esqnonio, -0.02cm) rectangle (\esqnonio + \largnonio,-0.5cm);
      
    \foreach \x in {0,...,10}
        \draw (\x*0.19cm, -0.01cm) -- (\x*0.19cm, -0.15cm)
        node[anchor=north] {{\tiny \x}};

    \foreach \x in {0,...,9}
        \draw (\x*0.19cm + 0.095cm, -0.01cm) -- (\x*0.19cm + 0.095cm, -0.10cm)
        node[anchor=south] {};
        
	\draw [fill] (0*0.19cm , -0.15cm) circle (0.02);
    
\end{tikzpicture}
\\

Precisão:\rule{3cm}{0.4pt}
\\
\vspace{1cm}

\end{minipage}
\begin{minipage}{0.4\linewidth}

\def\esqnonio{0.43cm}
\begin{tikzpicture}
%régua
    \draw (0,0) -- (3cm,0);
    
    \foreach \x in {0,...,2}
        \draw (\x*1cm,0pt) -- (\x*1cm,15pt)
        node[anchor=south] {$\x$};

    \foreach \x in {1,...,29}
        \draw (\x*0.1cm,0pt) -- (\x*0.1cm,5pt)
        node[anchor=south] {};

    \foreach \x in {1,3,5}
        \draw (\x*0.5cm,0pt) -- (\x*0.5cm,10pt)
        node[anchor=south] {};

%Nônio
	\draw [gray, ultra thick, fill=lightgray] (\esqnonio, -0.03cm) rectangle (\esqnonio + \largnonio,-0.5cm);
      
    \foreach \x in {0,...,10}
        \draw (\esqnonio + 0.0975cm + \x*0.19cm, -0.01cm) -- (\esqnonio + 0.0975cm + \x*0.19cm, -0.15cm)
        node[anchor=north] {{\tiny \x}};

    \foreach \x in {0,...,9}
        \draw (\esqnonio + 0.095cm + \x*0.19cm + 0.095cm, -0.01cm) -- (\esqnonio + 0.095cm + \x*0.19cm + 0.095cm, -0.10cm)
        node[anchor=south] {};

	\draw [fill] (\esqnonio + 0.095cm + 2.5*0.19cm, -0.10cm) circle (0.02);

\end{tikzpicture}
\\

Precisão:\rule{3cm}{0.4pt}
\\
\vspace{1cm}
\end{minipage}
\begin{minipage}{0.4\linewidth}

\def\esqnonio{0.59cm}
\begin{tikzpicture}
%régua
    \draw (0,0) -- (3cm,0);
    
    \foreach \x in {0,...,2}
    	\def\y{\x + 13}
        \draw (\x*1cm,0pt) -- (\x*1cm,15pt)
        node[anchor=south] {$\directlua{tex.print(\y);}$};

    \foreach \x in {1,...,29}
        \draw (\x*0.1cm,0pt) -- (\x*0.1cm,5pt)
        node[anchor=south] {};

    \foreach \x in {1,3,5}
        \draw (\x*0.5cm,0pt) -- (\x*0.5cm,10pt)
        node[anchor=south] {};

%Nônio
	\draw [gray, ultra thick, fill=lightgray] (\esqnonio, -0.03cm) rectangle (\esqnonio + \largnonio,-0.5cm);
      
    \foreach \x in {0,...,10}
        \draw (\esqnonio + 0.0975cm + \x*0.19cm, -0.01cm) -- (\esqnonio + 0.0975cm + \x*0.19cm, -0.15cm)
        node[anchor=north] {{\tiny \x}};

    \foreach \x in {0,...,9}
        \draw (\esqnonio + 0.095cm + \x*0.19cm + 0.095cm, -0.01cm) -- (\esqnonio + 0.095cm + \x*0.19cm + 0.095cm, -0.10cm)
        node[anchor=south] {};

	\draw [fill] (\esqnonio + 0.095cm + 8.5*0.19cm, -0.10cm) circle (0.02);

\end{tikzpicture}
\\

Precisão:\rule{3cm}{0.4pt}
\\
\vspace{1cm}

\end{minipage}
\begin{minipage}{0.4\linewidth}

\def\esqnonio{0.27cm}
\begin{tikzpicture}
%régua
    \draw (0,0) -- (3cm,0);
    
    \foreach \x in {0,...,2}
    	\def\y{\x + 13}
        \draw (\x*1cm,0pt) -- (\x*1cm,15pt)
        node[anchor=south] {$\directlua{tex.print(\y);}$};

    \foreach \x in {1,...,29}
        \draw (\x*0.1cm,0pt) -- (\x*0.1cm,5pt)
        node[anchor=south] {};

    \foreach \x in {1,3,5}
        \draw (\x*0.5cm,0pt) -- (\x*0.5cm,10pt)
        node[anchor=south] {};

%Nônio
	\draw [gray, ultra thick, fill=lightgray] (\esqnonio, -0.03cm) rectangle (\esqnonio + \largnonio,-0.5cm);
      
    \foreach \x in {0,...,10}
        \draw (\esqnonio + 0.0975cm + \x*0.19cm, -0.01cm) -- (\esqnonio + 0.0975cm + \x*0.19cm, -0.15cm)
        node[anchor=north] {{\tiny \x}};

    \foreach \x in {0,...,9}
        \draw (\esqnonio + 0.095cm + \x*0.19cm + 0.095cm, -0.01cm) -- (\esqnonio + 0.095cm + \x*0.19cm + 0.095cm, -0.10cm)
        node[anchor=south] {};

	\draw [fill] (\esqnonio + 0.095cm + 6.5*0.19cm, -0.10cm) circle (0.02);

\end{tikzpicture}
\\

Precisão:\rule{3cm}{0.4pt}

\vspace{1cm}
\end{minipage}
\begin{minipage}{0.4\linewidth}

\def\esqnonio{0.27cm}
\begin{tikzpicture}
%régua
    \draw (0,0) -- (3cm,0);
    
    \foreach \x in {0,...,2}
    	\def\y{\x + 13}
        \draw (\x*1cm,0pt) -- (\x*1cm,15pt)
        node[anchor=south] {$\directlua{tex.print(\y);}$};

    \foreach \x in {1,...,29}
        \draw (\x*0.1cm,0pt) -- (\x*0.1cm,5pt)
        node[anchor=south] {};

    \foreach \x in {1,3,5}
        \draw (\x*0.5cm,0pt) -- (\x*0.5cm,10pt)
        node[anchor=south] {};

%Nônio
	\draw [gray, ultra thick, fill=lightgray] (\esqnonio, -0.03cm) rectangle (\esqnonio + \largnonio,-0.5cm);
      
    \foreach \x in {0,...,10}
        \draw (\esqnonio + 0.0975cm + \x*0.19cm, -0.01cm) -- (\esqnonio + 0.0975cm + \x*0.19cm, -0.15cm)
        node[anchor=north] {{\tiny \x}};

    \foreach \x in {0,...,9}
        \draw (\esqnonio + 0.095cm + \x*0.19cm + 0.095cm, -0.01cm) -- (\esqnonio + 0.095cm + \x*0.19cm + 0.095cm, -0.10cm)
        node[anchor=south] {};

	\draw [fill] (\esqnonio + 0.095cm + 6.5*0.19cm, -0.10cm) circle (0.02);

\end{tikzpicture}
\\

Precisão:\rule{3cm}{0.4pt}
\end{minipage}
\end{minipage}

\section{Procedimento}
Obs:Antes de você fazer esta prática é conveniente conhecer o conteúdo do texto sobre \emph{Algarismos Significativos}. O aluno que não observar as regras sobre Algarismos Significativos em seus relatórios será penalizado.

\subsection{Cálculos de volumes e diâmetros}
Utilizando o cálculo do \emph{valor médio},em que o número de termos  é o mesmo dos números componentes da equipe, como uso do paquímetro, determine:

\subsubsection{O volume da peça cilíndrica maior} 
\begin{table}[h]
\centering
\begin{tabular}{|l|*{4}{c|}}
\hline & Medida& Medida& Medida& Medida \\
\hline Diâmetro(mm)& & &  & \\ 
\hline Altura(mm)& & &  & \\ 
\hline
\end{tabular}

\end{table}

\begin{table}[h]
\centering
\begin{tabular}{|p{10cm}|}
\hline Cálulo do Volume \\ 
\\
\\
\hline
\end{tabular}
\end{table}

\subsubsection{O diâmetro do tarugo} 
\begin{table}[h]
\centering
\begin{tabular}{|l|*{4}{c|}}
\hline & Medida& Medida& Medida& Medida \\
\hline Diâmetro(mm)& & &  & \\ 
\hline
\end{tabular}

\end{table}

\subsubsection{O volume de ferro da peça com furo cego} 

\begin{table}[ht]
\centering
\begin{tabular}{|l|*{4}{c|}}
\hline & Medida& Medida& Medida& Medida \\
\hline Diâmetro externo(mm)& & &  & \\ 
\hline Altura externa(mm)& & &  & \\ 
\hline Diâmetro interno(mm)& & &  & \\ 
\hline Altura interna(mm)& & &  & \\ 
\hline
\end{tabular}

\end{table}

\begin{table}[h]
\centering
\begin{tabular}{|p{10cm}|}
\hline Cálulo do Volume \\ 
\\
\\
\hline
\end{tabular}
\end{table}

\subsection{Outros cálculos}
Com o auxílio de tiras de papel, envolva as peças e, com uma régua, meça os comprimentos das circunferências externas. Anote somente os valores obtido por você.

\begin{table}[h]
\centering
\begin{tabular}{|p{10cm}|}
\hline 
\\
\\
\hline
\end{tabular}
\end{table}

\section{Questionário}
\begin{enumerate}
\item A partir dos valores médios dos diâmetros obtido com o paquímetro, determine o comprimento da circunferência externa das três peças.
\item Considere os valores dos comprimentos das circunferências obtidas com o paquímetro e com uma régua, quais os de maior precisão?
\item Nas medidas feitas na peça como o furo cego, para o cálculo do volume, quais as que podem contribuir no mesmo resultado com maior erro? Por quê?
\item Qual a menor fração de milímetro que pode ser lida com o paquímetro que você utilizou?
\item Qual a precisão de um paquímetro cujo nônio tem $49mm$ de comprimento e está dividido em 50 partes iguais?
\item O nônio de um paquímetro tem $29mm$ de comprimento. A precisão do mesmo é de $0,1mm$. En quantas partes foi dividido o nônio?
\item Num paquímetro de $0,05mm$ de sensibilidade, a distância  entre o zero da escala e o zero do vernier é de $11,5cm$, sendo que o 13º traço do vernier coincidiu. Qual o valor da medida?
\item Qual seria a leitura acima se a sensibilidade fosse $0,02mm$?
\end{enumerate}


\chapter[Análise Gráfica e MRUA]{Análise Gráfica e Movimento Retilíneo Uniformemente Acelerado}

\section{Introdução}
Um movimento retilíneo chama-se uniformemente acelerado quando a a\-ce\-le\-ra\-ção instantânea é constante (independente do tempo). Isto é,

\begin{equation}
\label{eq:aceleracao}
\dfrac{dv}{dt} = \dfrac{d^{2}x}{dt^2} =  a = constante
\end{equation}

Da \refeq{eq:aceleracao} na página \pageref{eq:aceleracao} podemos obter a equação horária da velocidade, que é dada por:

\begin{equation}
\label{eq:velocidade}
v(t) - v(t_0) = \int_{t_0}^{t}\!\!\!adt = a(t-t_0)
\end{equation}

O valor $v(t) = v(t_0)$ da velocidade no ins\-tan\-te inicial chama-se \emph{velocidade inicial}. Assim, $v(t) = v_0 + a(t-t_0)$ mostrando que a velocidade é uma função linear do tempo no movimento uniformemente acelerado.

Podemos obter a lei horária da posição integrando a equação da velocidade em função do tempo (\refeq{eq:velocidade}).

\begin{equation}
x(t) - x(t_0) = \int_{t_0}^{t}v(t')\!\!\!dt' = v_0(t - t_0) + \frac{1}{2}a(t - t_0)^2
\end{equation}

Se definirmos $x(t_0) = x_0$ como posição inicial. Obtemos, desta forma:

\begin{equation}
x(t) = x(t_0) + v_0(t-t_0) + \frac{1}{2}a(t - t_0)^2
\end{equation}

Também podemos exprimir a velocidade do movimento uniformemente acelerado em função da posição por $v^2 = v_{0}^2 + 2a(t-t_0)^2$; também conhecidad como equaçõa de Torricelli.

Aa esquaç~oes cima descrevem apenas a cinemática do movimento uniformemente acelerado, sem ter a preocupação de descrver a origem destes moviemntos - que é o objeto de estudo da dinâmica,cujos princípios básicos  forma formulador por Galileu e Newton.
\section{Parte Experimental}

\subsection{Objetivo}
analisar o movimento de um objeto sob a ação de uma força constante. Utilizar também a análise gráfica para descrever este movimento e determinar sua aceleração.

\subsection{Material Utilizado}

Talvez esta seja a primeira vez que você lida com um trilho de ar, assim, algumas notas de cuidado são úteis. O trilho possui pequenos orifícios pelos quais ar é expelido sob pressão. O carro que corre sobre o trilho tem o formato de um Y invertido, e se mantém flutuando sobre o colchão de ar formado entre o trilho e o carro pelo ar expelido nos cilindros. Assim, é eseencial manter os orifícios e a superfície do carro limpos e livre de arranhões. Evite, portanto, escrever ou marcar o trilho de ar para não obstruir os orifícios e causar variações no colchão de ar formado.

Importante: Não empurre o carrinho sobre o trilho quando a fonte de ar comprimido estiver ligada. Do contrário, tanto o carrinho quanto o trilho poderão sofrer arranhões.

O trilho de ar possui uma escala milimetrada que pode ser usada para registrar a posição do carro, e dispões de um cronômetro digital para registrar os intervalos de tempo. É mais simples com este equipamento medir o tempo transcorrido em função da distância a ser percorrida, embora posteriormente você possa inverter a dependência e analisar a posição em função do tempo transcorrido.

Decrição do material:

\begin{itemize}
\item 01 trilho de 12cm;
\item cronômetro digital multifunções com fonte DV 12V;
\end{itemize}


\subsection{Procedimentos}

\section{Sugestão para condução da análise dos dados:}



\chapter{Referências}

Anote aqui todas as referência que utilizar e depois passamos para o formato bibtex.



\end{document}