\chapter*{Introdução}
A disciplina de laboratório de física IV tem como objetivo abordar tópicos experimentais relacionados à
disciplina de física IV. Nessa disciplina o estudante tem os primeiros contatos com experiências relacionadas ao
estudo de correntes alternadas, óptica  e física moderna. Na  medida do possível,  as experiências seguem a
mesma ordem da disciplina teórica de física IV. Espera-se com isso, que o estudante tenha a oportunidade de
entender o fenômeno físico do ponto  de vista teórico e experimental. A preparação  dos relatórios de cada
experiência deverá  seguir um padrão que permita  ao estudante entender o desenvolvimento do  método
científico. 

A disciplina de laboratório de física IV é uma matéria experimental, na qual a turma de estudantes se divide em 
grupos de trabalho. No início  de  cada aula, o  professor apresenta uma breve discussão teórica  sobre a
experiência que será realizada. Nessa discussão, os grupos também são orientados na seqüência lógica do 
procedimento experimental. Sugere-se que uma experiência completa deve ser executada em cada aula.  

