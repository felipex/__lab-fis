%
% exemplo.tex - Exemplo de uso do template abntex-utp.
%
% Note que o arquivo em que você deve editar seu trabalho é "trabalho.tex".
% É lá que você deve editar o texto do seu trabalho. Use este arquivo como
% referência de como implementar equações, tabelas quadros e referências.
%
% As referências bibliográficas (referenciadas com \ref{NOME}), ficam no
% arquivo "biblio.bib".
%
\documentclass[font=plain,chapter=TITLE,section=Title,espaco=duplo,tocpage=plain,appendix=Name,floatnumber=continuous]{abnt}
\usepackage{hyperref}
\usepackage[utf8]{inputenc}
\usepackage[brazil]{babel}
\usepackage[alf]{abntcite}
\usepackage{graphicx}

%% xunxo para seguir as normas da UTP
\usepackage{xunxos-utp}

%% informações sobre o trabalho
\autor{Nicolai Nicolaiev}
\coautor{Fulano Fulanowsky}
\titulo{Um exemplo de trabalho em \LaTeX{}}
\comentario{Trabalho de Conclusão de Curso apresentado ao Curso de Bacharelado
em Ciência da Computação, da Faculdade de Ciências Exatas da Universidade
Tuiuti do Paraná, como requisito parcial para a obtenção do grau de Bacharel em
Ciência da Computação.}
\instituicao{Universidade Tuiuti do Paraná}
\orientador[Orientador: ]{Parararan Parararanavam}
\local{Curitiba}
\data{2010}

\begin{document}
\input{xunxos-utp-doc.tex}

%
% comandos para gerar capas
%
\UTPCapa
\UTPFalsaFolhaDeRosto
\UTPFolhaDeRosto

\begin{resumo}
Este documento é um exemplo de uso do modelo \emph{abntex-utp}, que
consiste um template latex para as normas da Universidade Tuiuti do Paraná.
São apresentados vários usos comuns em trabalhos acadêmicos e TCC, como
equações, tabelas, quadros, figuras, referências e referências
bibliográficas.

Palavras-chave: fermento; sal; açúcar; linhaça
\end{resumo}

%
% comandos para gerar lista de figuras, lista de tabelas, lista de quadros.
% Caso não hajam tabelas, quadros ou figuras é só apagar a respectiva
% linha.
%
\listoffigures
\listoftables
\listadequadros
\sumario

%
% Agora sim, começa o conteúdo propriamente dito.
%
% Os tópicos de nível maior são feitos com o comando \chapter{}, os de
% nível menor são feitos com \section{}, e os de menor ainda são com
% \subsection{}. Você pode usar até \subsubsection{}.
%

\chapter{Introdução}

Este documento é uma demonstração do formato utilizado pelo pacote
\href{http://abntex-utp.googlecode.com}{ABNTeX-UTP}. Para mais detalhes,
veja \url{http://abntex-utp.googlecode.com}.

As experiências acumuladas demonstram que a contínua expansão de nossa
atividade afeta positivamente a correta previsão das novas proposições. Do
mesmo modo, a valorização de fatores subjetivos causa impacto indireto na
reavaliação dos paradigmas corporativos. Percebemos, cada vez mais, que o
fenômeno da Internet ainda não demonstrou convincentemente que vai
participar na mudança das diretrizes de desenvolvimento para o futuro. A
prática cotidiana prova que a determinação clara de objetivos talvez venha
a ressaltar a relatividade das formas de ação. 

\chapter{Revisão da Literatura}

%
% Alguns exemplos de citação. \cite{} cria uma citação na forma
% (SOBRENOME, ANO).
%
% \citeonline cria uma citação direta, que fica na forma
% Sobrenome (ANO)
%
% Esses textos como "joachims1998text" são a chave da citação, estas chaves
% são definidas em biblio.bib, por exemplo:
%
%    @conference{joachims1998text,
%      title={{Text categorization with support vector machines: learning with many relevant features}},
%      author={Joachims, T. and Nedellec, C. and Rouveirol, C.},
%      booktitle={Machine Learning: ECML-98 10th European Conference on Machine Learning},
%      pages={137--142},
%      year={1998},
%      address={Chemnitz, Alemanha},
%      organization={Springer}
%    }
%
% Estas informações (title, author, pages, year, etc) são usadas para gerar
% a entrada na página de referências. Essa página é gerada automaticamente
% e só as referências que você citou com \cite{} ou \citeonline{} são
% realmente utilizadas.
%
Segundo~\cite{joachims1998text}, a natureza do aprendizado estatístico não
deve-se apenas aos problemas da humanidade.

Segundo o sempre sábio \citeonline{umExemploDeCitacaoUsandoPaginasWeb},
alguma coisa disso pode ser verdade.

Porém, segundo~\cite{vapnik2000nature}, nada disso é verdade.

Além disso, é importante ter tópicos:
\begin{itemize}
  \item este é um item;
  \item este é outro; e
  \item este é mais um.
\end{itemize}

Pensando mais a longo prazo, a expansão dos mercados mundiais pode nos
levar a considerar a reestruturação de todos os recursos funcionais
envolvidos. Acima de tudo, é fundamental ressaltar que o aumento do diálogo
entre os diferentes setores produtivos auxilia a preparação e a composição
das diversas correntes de pensamento. Todavia, a execução dos pontos do
programa acarreta um processo de reformulação e modernização da gestão
inovadora da qual fazemos parte. Ainda assim, existem dúvidas a respeito de
como a necessidade de renovação processual prepara-nos para enfrentar
situações atípicas decorrentes dos modos de operação convencionais. A nível
organizacional, a competitividade nas transações comerciais exige a
precisão e a definição de alternativas às soluções ortodoxas.

E também é importante definir alguns passos:
\begin{enumerate}
  \item Nascer,
  \item Crescer, e
  \item Morrer.
\end{enumerate}

%
% Agora exemplos de equações matemáticas. Há mais em
% http://en.wikibooks.org/wiki/LaTeX/Mathematics
%
%
E agora, uma equação:
\begin{equation}
% um somatório:
\alpha = \sum_{i=0}^{N}{x_i + k^n}
\end{equation}

E em imagens, porém,~\cite{semolini2002support} afirma que generalização
para duas dimensões para uma matriz $p$ de tamanho $n \times n$:
\begin{equation}
G_{ij} = \frac{1}{\sqrt{2n}} C_i C_j \sum_{x=0}^{n-1} \sum_{y=0}^{n-1}
         p_{xy} \cos{ \left ( \frac{(2y + 1) j \pi}{2n} \right ) }
              \cos{ \left ( \frac{(2x + 1) i \pi}{2n} \right ) }
\label{eq:dct_duas}
\end{equation}

%
% Um exemplo de uma seção. Uma seção fica abaixo de um capítulo, então sua
% numeração sempre é 1.1, 1.2, 3.3, etc.
%
\section{Este é o título desta seção}

Caros amigos, o comprometimento entre as ontologias apresenta tendências no
sentido de aprovar a manutenção das múltiplas direções do ponto de
transcendência do sentido enunciativo. Por outro lado, a complexidade dos
estudos efetuados cumpre um papel essencial na formulação do gênio grego
fundado na poesia homérica. Assim como indicado dela definição
\ref{eq:dct_duas}.  Assim mesmo, a estrutura atual da ideação semântica
pode nos levar a considerar a o \textit{merge} na \textbf{reestruturação}
da definição espinosista de substância. Neste sentido, existem duas
tendências que coexistem de modo heterogêneo, revelando o novo modelo
estruturalista aqui preconizado auxilia a preparação e a composição das
posturas dos filósofos divergentes com relação às suas atribuições.
Contudo, a crítica contundente de Deleuze/Guatarri - dupla implacável - nos
mostra que a indeterminação contínua de distintas formas de fenômeno
garante a contribuição de um grupo importante na determinação das novas
teorias propostas.

%
% Um exemplo de figura no texto.
%
% Para alterar o tamanho dela, mude o "0.5" para outro valor. É uma escala
% de >0 a 1 em relação à largura da página.
%
% Recomendo que use arquivos .PNG "bem grandes", para garantir que fique
% bom.
%
\begin{figure}[h!]
  \centering
  \includegraphics[width=0.5\textwidth]{img/eagle.jpg}
  \figinfox{Uma águia e seu amigo.}{Wikipédia Commons, 2010}
\end{figure}

O incentivo ao avanço tecnológico, assim como a hegemonia do ambiente
político não pode mais se dissociar das condições inegavelmente
apropriadas. Gostaria de enfatizar que a constante divulgação das
informações oferece uma interessante oportunidade para verificação dos
relacionamentos verticais entre as hierarquias. O empenho em analisar a
mobilidade dos capitais internacionais é uma das consequências dos métodos
utilizados na avaliação de resultados \cite{semolini2002support}.

%%
%% Mais informações a respeito de tabelas em latex:
%% http://en.wikibooks.org/wiki/LaTeX/Tables
%%
%% Grandes dicas:
%% - "|l|l|l|" representa a formatação de cada coluna, sendo "l" de "left"
%%   (alinhado à esquerda) e o "|" significa que deve ter uma linha
%%   vertical
%% - & é o caractere que indica "tabulação", e no caso das tabelas, separa as colunas
%%
\begin{table}[h!b!p!]
\centering
\begin{tabular}{|l|l|l|}
\hline
Alphabet Character & Vowel & Number \\
\hline
A & Yes & 1 \\
B & No & 2 \\
C & No & 3 \\
\hline
\end{tabular}
\caption{Uma super tabela}
\label{tab:seila} % veja a descrição abaixo
\end{table}

%
% Note esse \ref{} no parágrafo abaixo. São referências cruzadas. Cada
% tabela, quadro, imagem vai ganhar um número. Para saber qual o número é
% usado esse comando \ref{}, assim no texto final ficaria "na tabela 10, a
% crescente".
%
% Se você precisa indicar em qual página estará algum elemento, você pode
% usar o comando \pageref{}, assim, poderia ficar "na tabela da página 4"
%
% Esses objetos ganham a numeração usando o comando \label{PREFIXO:NOME}.
% PREFIXO é o que indica o tipo do objeto, tab para tablea, fig para
% figura, quadro para quadro e sec pra seções ou capítulos (\section{} ou
% \chapter{}).
% NOME é qualquer apelido que você quiser dar para o objeto.
%
De acordo com as idéias de Deleuze como apresentado na tabela \ref{tab:seila},
a crescente influência da mídia acarreta um processo de
reformulação e modernização das diversas correntes de pensamento.  Como
afirmou Deleuze, o Übermensch de Nietzsche, ou seja, o Super-Homem, exige a
precisão e a definição das ciências discursivas.  Pode-se argumentar, como
Bachelard fizera, que a teoria de Fliess é uma das consequências do
processo de comunicação como um todo. Como descrito na tabela da
página \pageref{tab:seila}.

%
% Um exemplo de uma tabela
%
\begin{table}[h!b!p!]
\centering
\begin{tabular}{lll} % cada l indica uma coluna na tabela
\hline % significa uma linha horizontal
Alphabet Character & Vowel & Number \\ % essas \\ indicam uma linha nova
\hline
A & Yes & 1 \\ % cada & indica a separação de uma coluna
B & No & 2 \\
C & No & 3 \\
\hline
\end{tabular}
\quadro{Eita} % aqui deve ficar uma descrição do quadro
\label{quadro:outrola}
\end{table}

Nunca podemos deixar de citar o velho mestre:

\begin{citacao}
O incentivo ao avanço tecnológico, assim como a hegemonia do ambiente
político não pode mais se dissociar das condições inegavelmente
apropriadas. Gostaria de enfatizar que a constante divulgação das
informações oferece uma interessante oportunidade para verificação dos
relacionamentos verticais entre as hierarquias. O empenho em analisar a
mobilidade dos capitais internacionais é uma das consequências dos métodos
utilizados na avaliação de resultados \cite{semolini2002support}.
\end{citacao}

Enfim.

\begin{table}
\centering
\begin{tabular}{lll}
\hline
Alphabet Character & Vowel & Number \\
\hline
A & Yes & 1 \\
B & No & 2 \\
C & No & 3 \\
\hline
\end{tabular}
\quadro{Mais um problema}
\end{table}

\subsection{Mais Alguma Coisa}

O incentivo ao avanço tecnológico, assim como a hegemonia do ambiente político
não pode mais se dissociar das condições inegavelmente apropriadas. Gostaria de
enfatizar que a constante divulgação das informações oferece uma interessante
oportunidade para verificação dos relacionamentos verticais entre as
hierarquias. O empenho em analisar a mobilidade dos capitais internacionais é
uma das consequências dos métodos utilizados na avaliação de resultados. Os
resultados podem ser observados no quadro \ref{quadro:outrola}.

\subsubsection{E uma subsubseção}

O incentivo ao avanço tecnológico, assim como a hegemonia do ambiente político
não pode mais se dissociar das condições inegavelmente apropriadas. Gostaria de
enfatizar que a constante divulgação das informações oferece uma interessante
oportunidade para verificação dos relacionamentos verticais entre as
hierarquias. O empenho em analisar a mobilidade dos capitais internacionais é
uma das consequências dos métodos utilizados na avaliação de resultados. Os
resultados podem ser observados no quadro \ref{quadro:outrola}.

\chapter{Implementação}

\begin{figure}[h!]
  \centering
  \includegraphics{img/taras.png}
  % note que width não é especificado aqui, a imagem vai ficar com o
  % tamanho máximo, veja seu uso na outra imagem
  \figinfox{Tarás Shevchenko}{Peguei Nanét e associados, 2010}
  \label{fig:shevchenko}
\end{figure}

Segundo o genial Heidegger, o entendimento dos universais antropológicos
ainda não demonstrou convincentemente como vai participar na mudança das
condições epistemológicas e cognitivas exigidas. Nietzsche diria que o
aumento do diálogo entre os diferentes setores filosóficos limita as
atividades das condições de suas incógnitas. Prospectos designam, de
início, a expansão dos mercados mundiais prepara-nos para enfrentar
situações atípicas decorrentes de todos os recursos funcionais envolvidos.
Como indica a figura \ref{fig:shevchenko}. Todas estas questões,
devidamente ponderadas, levantam dúvidas sobre se a hegemonia das
estruturas do poder repressivoé um subconjunto da corrente inovadora da
qual fazemos parte.

\begin{figure}[h!]
  \centering
  % note que a imagem é .pdf; para figuras que são gráficos, geralmente é
  % melhor tentar converter para .pdf antes para garantir mais definição na
  % hora de imprimir
  \includegraphics[width=0.3\textwidth]{img/fulanos.pdf}
  % note que é usado \figinfo{} em vez de \figinfox, pois a imagem é de
  % autoria própria e não precisa do FONTE: blablabla.
  \figinfo{Um gráfico simples exemplificando imagens de autoria própria}
  \label{fig:fulanos}
\end{figure}

Segundo o genial Heidegger, o entendimento dos universais antropológicos
ainda não demonstrou convincentemente como vai participar na mudança das
condições epistemológicas e cognitivas exigidas. Nietzsche diria que o
aumento do diálogo entre os diferentes setores filosóficos limita as
atividades das condições de suas incógnitas. Prospectos designam, de
início, a expansão dos mercados mundiais prepara-nos para enfrentar
situações atípicas decorrentes de todos os recursos funcionais envolvidos.
Como indica a figura \ref{fig:shevchenko}. Todas estas questões,
devidamente ponderadas, levantam dúvidas sobre se a hegemonia das
estruturas do poder repressivoé um subconjunto da corrente inovadora da
qual fazemos parte.

\section{Outra Coisa Finalmente}

Caros amigos, a crescente influência da mídia nos obriga à análise dos
procedimentos normalmente adotados. A certificação de metodologias que nos
auxiliam a lidar com o início da atividade geral de formação de atitudes
estende o alcance e a importância do retorno esperado a longo prazo. Assim
mesmo, a complexidade dos estudos efetuados assume importantes posições no
estabelecimento do sistema de participação geral.

\chapter{Conclusão}

Segundo~\cite{normasUTP}, a conclusão “trata-se do momento de apresentar,
de forma sintética, os resultados obtidos para o problema de pesquisa”.

\chapter{Considerações Finais}

Segundo~\cite{normasUTP}, “ao elaborar as considerações finais, o
pesquisador deve ter em mente o problema e os objetivos da pesquisa, de
forma que possa construir o texto elaborando respostas coerentes para os
mesmos”.

%
% As entradas de bibliográficas (as "referências") estão em biblio.bib.
%
\bibliography{biblio}

\end{document}
%% vim:ft=tex
