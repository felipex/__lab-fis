\documentclass[10pt,a4paper,onecolumn,notitlepage]{scrartcl}

%%  Os pacotes da AMS devem ser carregados antes de fontenc e babel (Dica do Uma não tão pequena introdução ao LATEX 2ε)
\usepackage{amsmath}  
\usepackage{amsfonts}
\usepackage{amssymb}

\usepackage[utf8x]{inputenc} % compilando com o lualatex não precisa desta linha. lualatex é unicode por padrão.
\usepackage[brazil]{babel}
\usepackage[T1]{fontenc} % sem esse pacote a hifenização não funciona corretamente com palavras acentuadas.
\usepackage{ucs}
\usepackage{graphicx}
%\usepackage{wrapfig}
\usepackage{mathpazo}
\usepackage{hyperref}
%\usepackage{layouts}
%\usepackage{scrpage2}
\usepackage{tikz}
\usepackage{multicol}
\usepackage{siunitx}
\usepackage{enumitem}
\usepackage{float}
\usepackage{fancyhdr}

\usepackage[textheight=24cm]{geometry}

%\renewcommand{\thesection}{\arabic{section}}
%\numberwithin{equation}{section}
\newcommand{\refeq}[1]{Equação \eqref{#1}}
\newcommand{\HRule}{\rule{\linewidth}{0.5mm}}

\author{}
\title{Prática 1: Paquímetro}


%\thispagestyle{myheadings}
\pagestyle{fancy}
\lhead{}
\chead{}
\rhead{\textsc{Pratica 1: Paquímetro}}
\cfoot{\thepage}
\renewcommand{\headrulewidth}{0.4pt}

%\setcounter{tocdepth}{1} % Profundidade do sumário: 0 - mostra apenas os capítulos.

\begin{document}
\thispagestyle{myheadings}

\begin{figure}
\begin{minipage}{0.08\linewidth}
\includegraphics[scale=0.5]{figuras/brasaoUFC.jpg} 
\end{minipage}
\begin{minipage}{0.91\linewidth}
\textsc{Universidade Federal do Ceará}

Disciplina: EM0016 - Física Experimental para Engenharia
\end{minipage}

\begin{minipage}{\linewidth}
%\vspace{0.1cm}
\centering
\textsc{Algarismos Significativos e Erros}
\\
\hrulefill % usar este comando quando não estiver dentro de uma tabela.
\end{minipage}
\end{figure}

\section{Algarismos Significativos e Erros}
Medir uma grandeza significa compará-la com outra de mesma espécie e verificar quantas vezes a primeira é menor ou maior do que esta.

Em geral, a precisão de uma medida é determinada pelo instrumento através do qual a medida é realizada e pela habilidade da pessoa que a realiza. Ao fazermos uma medida, devemos expressá-la de maneira que o resultado represente o melhor possível a grandeza medida. Por exemplo, ao medirmos o comprimento mostrado na Figura 1 com uma régua graduada em centímetros verificamos que o mesmo tem com certeza mais de $14cm$. Podemos estimar também que além dos $14cm$ temos mais uns $3mm$. Dizemos, então que o comprimento médido é $14,3cm$. Observe que nesta medida os algarismos 1 e 4 são exatos enquanto que o 3 foi estimado, sendo, portanto um algarismo duvidoso. Por que, então não expressamos o comprimento somente com  os algarismos corretos? A resposta é que $14,3$ dá uma melhor ideia do comprimento medido do que simplesmente $14cm$. Temos, então, 3 algarismos significativos.

Na Figura 2 podemos dizer que o valor medido é $14,35cm$, sendo os algarismos $1,4$ e $3$ todos corretos e o algarismo $5$ estimado. Neste caso temos uma medida com quatro algarismos significativos.

\textbf{Em uma medida, chamamos de algarismos significativos, todos os algarismos corretos mais o primeiro duvidoso.}

O algarismo duvidoso surge sempre ao estimarmos uma fração da menor divisão da escala do aparelho de medida.

\subsection{Operações com Algarismos Significativos}

\subsubsection{Soma e/ou Subtração}
\begin{itemize}
\item Algarismo correto $\pm$ algarismo correto = algarismo correto
\item Algarismo correto $\pm$ algarismo duvidoso = algarismo duvidoso
\item Algarismo duvidoso $\pm$ algarismo duvidoso = algarismo duvidoso
\end{itemize} 

Exemplo 1: Os lados de um triângulo foram medidos por instrumentos diferentes. Obteve-se os seguintes valores: $15,31cm$, $8,752cm$ e $17,7cm$. Calcule o perímetro.

\[
\begin{array}{rrlll}
& 15, & \!\!\!\!\!\!3\underline{1} &\\
+ & 8, & \!\!\!\!\!\!75\underline{2} & \text{* Os algarismos sublinhados são os algarismos duvidosos.}\\
& 17, & \!\!\!\!\!\!\underline{7}  & \\
\hline
& 41, & \!\!\!\!\!\!\underline{762} &
\end{array}
\]

O resultado deve conter apenas uma algarismo duvidoso; portanto $41,8cm$. (Observer a regra do arredondamento: \textbf{Se o algarismo à direita do menor Algarismo Significativo na resposta final é 4 ou menor, o valor é arredondado para baixo. Se o algarismo à direita do menor Algarismo Significativo na resposta final é 5 ou maior, o valor é arredondado para cima}).

O menor algarismo significativo de um número é aquele mais à direita.

Exemplo 2: Subtraia $46,7g$ de $96g$.

\[
\begin{array}{rrlll}
& 9\underline{6} &  &\\
+ & 46, & \!\!\!\!\!\!\underline{7} & \text{* Os algarismos sublinhados são os algarismos duvidosos.}\\
\hline
& 4\underline{9}, & \!\!\!\!\!\!\underline{3} &
\end{array}
\]



\end{document}