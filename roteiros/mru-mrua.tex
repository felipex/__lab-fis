\chapter[Análise Gráfica e MRUA]{Análise Gráfica e Movimento Retilíneo Uniformemente Acelerado}

\section{Introdução}
Um movimento retilíneo chama-se uniformemente acelerado quando a a\-ce\-le\-ra\-ção instantânea é constante (independente do tempo). Isto é,

\begin{equation}
\label{eq:aceleracao}
\dfrac{dv}{dt} = \dfrac{d^{2}x}{dt^2} =  a = constante
\end{equation}

Da \refeq{eq:aceleracao} na página \pageref{eq:aceleracao} podemos obter a equação horária da velocidade, que é dada por:

\begin{equation}
\label{eq:velocidade}
v(t) - v(t_0) = \int_{t_0}^{t}\!\!\!adt = a(t-t_0)
\end{equation}

O valor $v(t) = v(t_0)$ da velocidade no ins\-tan\-te inicial chama-se \emph{velocidade inicial}. Assim, $v(t) = v_0 + a(t-t_0)$ mostrando que a velocidade é uma função linear do tempo no movimento uniformemente acelerado.

Podemos obter a lei horária da posição integrando a equação da velocidade em função do tempo (\refeq{eq:velocidade}).

\begin{equation}
x(t) - x(t_0) = \int_{t_0}^{t}v(t')\!\!\!dt' = v_0(t - t_0) + \frac{1}{2}a(t - t_0)^2
\end{equation}

Se definirmos $x(t_0) = x_0$ como posição inicial. Obtemos, desta forma:

\begin{equation}
x(t) = x(t_0) + v_0(t-t_0) + \frac{1}{2}a(t - t_0)^2
\end{equation}

Também podemos exprimir a velocidade do movimento uniformemente acelerado em função da posição por $v^2 = v_{0}^2 + 2a(t-t_0)^2$; também conhecidad como equaçõa de Torricelli.

Aa esquaç~oes cima descrevem apenas a cinemática do movimento uniformemente acelerado, sem ter a preocupação de descrver a origem destes moviemntos - que é o objeto de estudo da dinâmica,cujos princípios básicos  forma formulador por Galileu e Newton.
\section{Parte Experimental}

\subsection{Objetivo}
analisar o movimento de um objeto sob a ação de uma força constante. Utilizar também a análise gráfica para descrever este movimento e determinar sua aceleração.

\subsection{Material Utilizado}

Talvez esta seja a primeira vez que você lida com um trilho de ar, assim, algumas notas de cuidado são úteis. O trilho possui pequenos orifícios pelos quais ar é expelido sob pressão. O carro que corre sobre o trilho tem o formato de um Y invertido, e se mantém flutuando sobre o colchão de ar formado entre o trilho e o carro pelo ar expelido nos cilindros. Assim, é eseencial manter os orifícios e a superfície do carro limpos e livre de arranhões. Evite, portanto, escrever ou marcar o trilho de ar para não obstruir os orifícios e causar variações no colchão de ar formado.

Importante: Não empurre o carrinho sobre o trilho quando a fonte de ar comprimido estiver ligada. Do contrário, tanto o carrinho quanto o trilho poderão sofrer arranhões.

O trilho de ar possui uma escala milimetrada que pode ser usada para registrar a posição do carro, e dispões de um cronômetro digital para registrar os intervalos de tempo. É mais simples com este equipamento medir o tempo transcorrido em função da distância a ser percorrida, embora posteriormente você possa inverter a dependência e analisar a posição em função do tempo transcorrido.

Decrição do material:

\begin{itemize}
\item 01 trilho de 12cm;
\item cronômetro digital multifunções com fonte DV 12V;
\end{itemize}


\subsection{Procedimentos}

\section{Sugestão para condução da análise dos dados:}


